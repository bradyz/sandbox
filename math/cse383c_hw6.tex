\documentclass[12pt]{article}

\usepackage{geometry}
\usepackage{amsmath}
\usepackage{amsthm}
\usepackage{amsfonts}
\usepackage{amssymb}
\usepackage{enumitem}
\usepackage{graphicx}
\usepackage{pdfpages}

\topmargin=-0.45in
\evensidemargin=0in
\oddsidemargin=0in
\textwidth=6.5in
\textheight=9.0in
\headsep=0.25in
\linespread{1}
\allowdisplaybreaks

\title{\textbf{CSE 383C:\@ Assignment 6}}
\author{Brady Zhou}
\date{\today}

\newenvironment{exercise}[2][Exercise]{\begin{trivlist}
\item[\hskip \labelsep{\bfseries #1}\hskip \labelsep{\bfseries #2.}]}{\end{trivlist}}

\begin{document}

\maketitle

\begin{exercise}{1} Find perturbations that maximize the error in the
  solution for $Ax = b$.

  \begin{enumerate}[label=(\alph*)]

    \item Find $\Delta A$, with $\|\Delta A\|_2 = \tau \|A\|_2$,
      which maximizes $\|\Delta x \|_2$, where $(A + \Delta)(x + \Delta x) = b$.

      Taking the SVD of A, we can see we have a decomposition of the form -

      \begin{center}
        {\Large $U$}
        $\begin{bmatrix}
          2 & 0 \\
          0 & 0.001 \\
          0 & 0 \\
        \end{bmatrix}$
        {\Large $V^T$}
      \end{center}

      We want to find $\|\Delta A\|_2$ such that $A + \Delta A$ spans a different
      space.
      This can be achieved by putting all the energy into the third row of $\Sigma$,
      since this $U_3$ is orthogonal to the range of A.
      $U$ and $V^T$ are orthonormal, they have a norm of $1$, so we can
      reduce the problem to finding a $\Delta \Sigma$ with the same constraints. \\
      Since we require $\|\Delta \Sigma\|_2 = \tau \|A\|_2$,
      and $\|A\|_2 = \sigma_1$, $\Delta A$ takes the form -

      \begin{center}
        {\Large $U$}
        $\begin{bmatrix}
          0 & 0 \\
          0 & 0 \\
          0 & -\tau \sigma_1 \\
        \end{bmatrix}$
        {\Large $V^T$}
      \end{center}

      Since $A$ and $\Delta A$ are still full rank, we can solve for
      $x$ and $\Delta x$ using the corresponding pseudoinverses, and the
      relative error comes out to $\sim17200\%$, a surprising result.

    \item Find $\Delta b$, with $\|\Delta b\|_2 = \tau \|b\|_2$, which maximizes
      $\|\Delta x\|_2$, where $A(x + \Delta x) = (b + \Delta b)$.

      From the properties of SVD, we know that $U_1$ and $U_2$ span the range of $A$.
      We want to pick a direction $U_i$ to ``push" $b$ further away from the range
      of $A$. \\
      Since $U_3$ is not in the range of $A$, there is no point in
      perturbing $b$ in that direction. $U_2$ is ``poorly spanned" by $A$, so
      this will be the best direction to perturb $b$ in. \\
      $\Delta b$ will be a scalar multiple $\alpha$ times $U_2$. \\
      The constraints on $\Delta b$ makes $\alpha = \tau \|b\|_2$, so
      $\Delta b = \tau \|b\|_2 U_2$. \\

      Solving for $x$ and $\Delta x$ using the pseudoinverses gives us that
      the relative error comes out to $\sim20\%$.

  \end{enumerate}

\end{exercise}

\pagebreak

\begin{exercise}{2} Let $A \in \mathbb{R}^{m \times n}$, $m < n$, rank($A$) $= m$.
  A is underdetermined and full rank. Let $b \in \mathbb{R}^m$.
  Want to find $x \in \mathbb{R}^n$ such that $Ax = b$.

  \begin{enumerate}[label=(\alph*)]

    \item Assume that $x = A^Ty$.

      Now we want to find the solution to $AA^Ty = b$. \\
      We can see $AA^T \in \mathbb{R}^{m \times m}$, and since rank($A$) $= m$,
      rank($AA^T$) $= m$, so $AA^T$ is full rank and square and we can take
      the inverse of this.

      \begin{center}
        $y = (A A^T)^{-1} b$ \\
        $x = A^T (A A^T)^{-1} b$ \\
      \end{center}

    \item Use the QR factorization of $A^T$.

      For QR decomposition, we have the requirement that $m \geq n$, so $A^T$
      satisfies this. The fact that $A$ is full rank allows us to use either
      Gram-Schmidt or Householder, but we have to make sure to used reduced QR. \\
      With a reduced QR, we have $A^T = QR$, where $Q \in \mathbb{R}^{n \times m}$,
      and $R \in \mathbb{R}^{m \times m}$, since $A$ is full rank. \\
      So we have $A = R^T Q^T$, and want to solve $R^T Q^T x = b$.
      Since $R^T$ is full rank and square, there exists an inverse $R^{-T}$. \\
      So now, $Q^T x = R^{-T} b$, and since the solution $x$ lives in the rowspace
      of $A$, and $Q^T$ forms a basis for the range of $A$, $Q$ forms a basis
      for the rowspace of $A$ and we have $x = Q y$. \\
      $ Q^T Q y = R^{-T} b$, which gives us that $y = R^{-T} b$.

      \begin{center}
        $y = R^{-T} b$ \\
        $x = Q R^{-T} b$ \\
      \end{center}

    \item Relation to TSVD.

      We can see finding a decomposition of the transpose and using that is
      equivalent to constructing the projector $U_k$ from the TSVD.

  \end{enumerate}

\end{exercise}

\begin{exercise}{3} Least squares problems when A is nearly rank deficient.

  \begin{enumerate}[label=(\alph*)]

    \item Plotting singular values.

      See attached.

    \item Rank analysis. Numerical rank is calculated with MATLAB
      rank($A$, $\sigma_1 * \tau$).

      \begin{center}

        \begin{tabular}{ | l | l | l | l | }
          \hline
          M & N & Rank r & Numerical Rank k \\ \hline
          1024 & 512 & 512 & 25 \\ \hline
          512 & 1024 & 512 & 21 \\ \hline
          1024 & 512 & 99 & 17 \\ \hline
          1024 & 512 & 512 & 44 \\ \hline
          1024 & 1024 & 1023 & 469 \\ \hline
        \end{tabular}

      \end{center}

      We can see for certain problems the matrix is clearly full rank. However,
      the numerical rank are quite low for many of the problems provided.

    \item Best k-rank approximation for range of A.

      We can see that SVD performs the best, but takes the longest.
      Both QR schemes (pivoted and nonpivoted) have unbounded amount of error,
      but with pivoting, we can see the relative error is less overall.
      The randorth scheme also predicts fairly well.

    \item Figure descriptions.

      The first five figures consist of plotting singular values.
      The next five figures consist of relative error for $x_k$ as a function of
      k-rank approximations using the TSVD.
      The next 5 figures consist of l-curve analysis for problems 1-5.

    \includepdf[pages=-,pagecommand={},width=\textwidth]{cse383c_matlab.pdf}
    \includepdf[pages=-,pagecommand={},width=\textwidth]{lcurve.pdf}


  \end{enumerate}

\end{exercise}

\end{document}
